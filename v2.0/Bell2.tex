\documentclass[prd,showpacs,twocolumn]{revtex4-1}
\usepackage{graphicx}% Include figure files
\usepackage{dcolumn}% Align table columns on decimal point
\usepackage{bm}% bold math
\usepackage{amsmath}
%\nofiles
\begin{document}
\title{Experiments testing Bell's inequality with local real source}
\author{Peifeng Wang}
\email{peifeng\_w@yahoo.com}
\address{Mijiaqiao 34-1-3-5, Xi'an, Shaanxi, P. R. China 710075}
\address{Guanghua Road 1\#, 34-1-3-5, Yanta District, Xi'an, Shaanxi, P. R. China 710075}
\begin{abstract}
In the 80th year since EPR paradox was proposed, 3 loophole free violation of Bell's inequality were presented in attempts to conclusively rule out local realism. But when the setup and the collected data in these reports are examined along with a) the physics concept of entanglement and b) precise interpretation of experiments, it shows that these tests were only performed on local real systems. Furthermore, neither locality nor fair sampling loophole was properly closed, not even individually. This analysis examines the wave function in various parts of the experiments, which sheds light on what actually happens in the experiments.
\end{abstract}
\pacs{03.65.Ud}
\maketitle
%249489

Bell's inequality\cite{Bell,CHSH}/entanglement has been extensively studied\cite{Aspect,Weihs, Rowe} as it distinguishes quantum mechanics(QM) from local real theories. In a typical setup in testing Bell's inequality, entangled particle pairs are measured by Alice and Bob respectively, and the correlation of the outcome are computed subsequently. Various experiments have been reported to violate Bell's inequality, in support of QM while ruling out local real theory.

But in addition to Bell's inequality, QM and local real theory have other aspects which are expected to be consistent with a valid experiment. In particular, one can examine the physics concept of entanglement and precise interpretation of the experiments. It is shown that 1) In a reported loophole-free violation of Bell inequality, the transition of wave function from odd parity to even parity reveals that the experiment is performed on the spin of a pair of local real nitrogen vacancy (NV) centre. 2) The equivalence between rotating spin by $\theta$ and rotating measurement basis by $-\theta$ is not applicable in entanglement case, thus in many long range experiments for closing locality loophole, the operations of rotating the spin followed by measurement introduce extra transition of wave function, therefore the subsequent measurements in these experiments are only taken on local real particles. 3) Fair sampling assumption arises when a finite sample is used to represent the entire population space, thus it is a basic requirement of statistical experiment, fair sampling loophole can not be closed.

Originally identified in EPR paradox \cite{EPR}, the physical concept of entanglement is specified with certainty that, for 2 entities once in interaction and later separated, impact on one entity will instantaneously change the other entity. The opposite concept, the local real model is that for 2 entities not in interaction, each can be changed independently without interfering with the other. Which one of these 2 mutually exclusive concepts properly describes the nature is the subject of extensive studies, including the derivation of Bell's inequality.

Aside from the obvious difference between the 2 physics concepts, there is also subtlety requiring precise interpretation. Due to relative motion, rotating the spin by $\theta$ is considered equivalent to rotating the measurement basis by $-\theta$. However, such equivalence is only valid in a local real model, where the impact of rotating a particle's spin is limited to the particle itself. In an entanglement case, rotating the particle's spin on the left side is supposed to 1) change the relative angle with respect to the measurement device on the left, 2) cause the entangled peer rotating its spin on the right. Thus in a presumed entanglement scenario, it is critical to precisely differentiate between a) rotating spin by $\theta$ and b) rotating measurement basis by $-\theta$.

In the reported loophole-free violation of Bell inequality \cite{Hensen}, the event ready theme only prepares entanglement in odd parity, in specific,
$\left| \psi^- \right>=(\left|\uparrow\downarrow\right>-\left|\downarrow\uparrow\right>)/\sqrt{2}$. So the prepared spin entanglement can only have correlation in $\uparrow\downarrow$ and $\downarrow\uparrow$ in +Z+Z \& -Z-Z projection.

But \cite{Hensen} also reports correlations in $\uparrow\uparrow$ and $\downarrow\downarrow$ in +Z-Z \& -Z+Z projection, as shown in Figure 3c therein. According to \cite{Hensen}, ``Readout in a rotated basis is achieved by first rotating the spin followed by readout along Z." i.e. Rather than rotating a measurement device, the actual operation of rotating the basis in \cite{Hensen} is to apply a microwave pulse to the NV centre to change the inherent state of the spin. As pointed out before, in an entanglement case, rotating spin by $\theta$ is different from rotating measurement basis by $-\theta$. Thus more precisely, data in -Z projection is read out in Z projection after rotating the spin by $\pi$, and the reported outcome in +Z-Z \& -Z+Z projection in \cite{Hensen} is actually result in +Z+Z \& -Z-Z projection after rotating spin by $\pi$ on one side.

Therefore, data in \cite{Hensen} sets up a situation to test the physics concept of entanglement/local real model directly. For the 2 cases in  Figure 3c of \cite{Hensen}, a) the upper left diagram shows the correlation in +Z+Z \& -Z-Z projection, with wave function in odd parity
\begin{eqnarray}
\left| \Psi_0\right >=\frac{1}{\sqrt{2}}(\left|\uparrow\downarrow\right>-\left|\downarrow\uparrow\right>)
\label{eqn:Psi_odd}
\end{eqnarray}
b) the lower left diagram shows the correlation in +Z-Z \& -Z+Z projection, which, in precise interpretation, translates to data in +Z+Z \& -Z-Z  projection after rotating spin by $\pi$ on one side, with wave function in even parity
\begin{eqnarray}
\left| \Psi_1\right >=\frac{1}{\sqrt{2}}(\left|\uparrow\uparrow\right>-\left|\downarrow\downarrow\right>)
\label{eqn:Psi_1}
\end{eqnarray}

Transition from $\left|\Psi_0\right>$ to $\left|\Psi_1\right>$ requires a change of spin state by $\pi$ on exactly one side, which agrees with the experimental operation of readout in +Z+Z \& -Z-Z projection after rotating spin by $\pi$ on one side. Observation of such transition shows that the spin on one side can be changed independently without impact on the other side, which is essentially a characteristics of local real model.

Although the widely held view is that only measurement on one of an entangled pair can change the other party, and rotating the spin of NV centre may not be considered a measurement, it needs to emphasize that in the experiment of \cite{Hensen}, rotating the spin of NV centre has measurable impact as manifested by the wave function transition from $\left|\Psi_0\right>$ to $\left|\Psi_1\right>$. Such change on one NV centre is supposed to have impact on the other if they are entangled. But as the transition from $\left|\Psi_0\right>$ to $\left|\Psi_1\right>$ indicates, rotating spin by $\pi$ on one side does not change the other side, it is a clear signature of local real model.

More generally, many experiments attempt to close the locality loophole by changing the measurement basis fast. These experiments \cite{Hensen,Giustina,Shalm} rely on the idea of relative motion, assuming rotating spin by $\theta$ as equivalent to rotating measurement basis by $-\theta$. However, as shown before, such equivalence assumption is only valid in local real model. In addition, rotating the spin and measurement are indeed two distinct steps (e.g. in the space-time analysis - Figure 2 of \cite{Hensen}, rotating the spin and measurement are 2 events). Thus in the precise description of these entanglement experiments, the spin of the particle is being changed by local real spin-rotating apparatus before actual measurement (rather than a single shot measurement with changing basis).

The operations of rotating spin followed by measurement introduce certain impact on the wave function. To determine the wave function on which Alice and Bob perform measurement, one can examine the transition of the wave function prior to the actual measurement, i.e. during the interaction with local real spin-rotating apparatus. If the wave function begins in entanglement,
\begin{eqnarray}
\left| \Psi\right >_{original}=\frac{1}{\sqrt{2}}(\left|\uparrow\downarrow\right>-\left|\downarrow\uparrow\right>)
\label{eqn:AnnotatedSinglet}
\end{eqnarray}
after interaction with the local real apparatus designed to change the particle state on both sides, the wave function must be different than (\ref{eqn:AnnotatedSinglet}) to reflect the change of spin due to the interactions. To meet the condition of free will, the interactions are arbitrarily configured by a controlling random signal, thus the wave function after interaction will contain all possible joint states, and becomes
\begin{eqnarray}
\left| \Psi\right >_{rotated}=\left|\uparrow\downarrow\right>+\left|\downarrow\uparrow\right> + \left|\uparrow\uparrow\right> + \left|\downarrow\downarrow\right>
\label{eqn:Disturbed}
\end{eqnarray}
the transition of wave function from $\left| \Psi\right >_{original}$ to $\left| \Psi\right >_{rotated}$ happens during interaction with spin rotating apparatus, before measurement. The result wave function $\left| \Psi\right >_{rotated}$ is in disentanglement, subsequently, Alice and Bob are only measuring disentangled particles.

$\left| \Psi\right >_{rotated}$ only arises in the operations of rotating spin followed by measurement. On the other hand, in a single shot measurement where the basis (rather than the spin) is rotated, there is no transition of wave function from $\left| \Psi\right >_{original}$ to $\left| \Psi\right >_{rotated}$ and the measurement is performed on the entanglement $\left| \Psi\right >_{original}$. Therefore, to close the locality loophole, an experiment shall implement single shot measurement with actual changing basis. For experiments assuming the equivalence of relative motion, operations of rotating spin followed by measurement only destroy entanglement before the measurement.

For another widely studied loophole, the one of fair sampling, many experiments attempt to close the issue by improving detector efficiency\cite{Giustina,Shalm,Hensen}. However, a precise understanding of fair sampling is still missing in the study of entanglement, and existing claims are inappropriate.

In fact, the fundamental principle of statistical experiment is to study the population through experimental samples, fair sampling assumption arises when finite experimental sample is used to represent the entire population. In theory, all inequalities must deal with some population parameters (e.g. probability $p_{++}$, correlation $C$), whereas in experiment, all tests can only collect raw count of events (e.g. $N_{++},N_{--}$).

Normally the population parameters in an inequality are estimated as statistics derived from the raw counts collected in an experiment, e.g.
\begin{eqnarray}
&&\hat{p}_{++}=\frac{N_{++}}{N_{++}+N_{--}+N_{+-}+N_{-+}}\nonumber\\
&&\hat{C}=\frac{N_{++}+N_{--}-N_{+-}-N_{-+}}{N_{++}+N_{--}+N_{+-}+N_{-+}}
\label{eqn:estimate}
\end{eqnarray}
the statistics are then used in place of corresponding population parameters in testing certain inequality. e.g.
\begin{eqnarray}
&&\hat{p}_{++}\rightarrow p_{++}\nonumber\\
&&\hat{C}\rightarrow C
\label{eqn:replace}
\end{eqnarray}
by definition, $p_{++}$ is the probability of event ++ in the entire population space, whereas $\hat{p}_{++}$ is the occurrence ratio of event ++ in a set of sample events from one particular experiment run. Since any set of experimental sample event is only a subset of the population space, the replacement (\ref{eqn:replace}) is using a subset to represent the entire population space.

Due to the infinite population space, a statistical experiment can not exhaust the entire population space, only finite samples of the population space are obtained. Thus when the finite sample is used to represent the population space in (\ref{eqn:replace}), one necessary assumption is that $\hat{p}_{++}$ derived from a finite set of sample events (i.e. a subset of entire population space) does properly represent $p_{++}$ of the entire population space, or alternatively, the sampling is fair.

As fair sampling assumption arises when a subset is used to represent the entire population space, it is independent of the detection efficiency in an experiment. Even if the detector efficiency is $100\%$, the collected experimental data set is still a subset of the population space, and fair sampling is still required when representing the population with a data set from a 100\% detector. Thus fair sampling loophole can not be closed, unless one is not doing a statistical experiment.

The fair sampling loophole originates from the statistical nature of Bell's inequality, which intends to test the local realism/entanglement hypothesis, whereas the underlying physics concept of entanglement (i.e. the subject to be tested) is specified with certainty. There has been extensive interest in studying Bell's inequality, as many experiments collected batches of data and computed the correlations after the test runs, though they do not attempt to directly test the concept of entanglement/local real model, i.e. to observe in real time whether change on one side has impact on the other side.

But the real fundamental question is about the physical essence of entanglement. If one could directly observe the spooky action at a distance, it would be a conclusive support of entanglement, which would leave test of Bell's inequality (by computing correlations after the happening moment of entanglement) superfluous. On the other hand, if an experiment has many aspects manifesting local realism (the opposite to entanglement), simple violation of Bell's inequality may not be effective.

As existing work show, transition of wave function from odd parity to even parity suggests that spin on one side can be independently changed without impact on the other side; operations of rotating spin followed by measurement disentangle the wave function to be measured; and as a basic requirement of representing the entire population with finite sample in any statistical experiment, fair sampling loophole can not be closed. Thus the reported experiments are more of a proof of local real model than evidence of entanglement.

In fact, what the reported experiments actually test includes every involved theory. Aside from entanglement and local realism, Bell's inequality is also a theoretical framework, whose practical effectiveness in distinguishing entanglement from local realism can not be taken for granted. In existing experiments intended to test local real model, the effectiveness of Bell's inequality may not be exempted from scrutiny.

\acknowledgments

\begin{thebibliography}{0}

\bibitem{Bell} J. S. Bell, On the Einstein Podolsky Rosen Paradox. Physics (Long Island City, N. Y.) 1, 195 (1965).
\bibitem{CHSH} J. F. Clauser, M. A. Horne, A. Shimony, and R. A. Holt, Proposed Experiment to Test Local Hidden-Variable
Theories. \emph{Phys. Rev. Lett.} 23, 880 (1969).
\bibitem{Aspect} A. Aspect, J. Dalibard, and G. Roger, Experimental Test of Bell's Inequalities Using Time-Varying Analyzers. Phys. Rev. Lett. 49, 1804 (1982).
\bibitem{Weihs} G. Weihs, T. Jennewein, C. Simon, H. Weinfurter, and A. Zeilinger, Violation of Bell's Inequality under Strict
Einstein Locality Conditions. Phys. Rev. Lett. 81, 5039 (1998).
\bibitem{Rowe} M. A. Rowe, D. Kielpinski, V. Meyer, C. A. Sackett, W. M. Itano, C. Monroe, and D. J. Wineland, Experimental violation of a Bell's inequality with efficient detection. Nature (London) 409, 791 (2001).
\bibitem{EPR} A. Einstein, B. Podolosky, and N. Rosen, Can Quantum-Mechanical Description of Physical Reality Be Considered Complete? Phys. Rev. 47, 777 (1935).
\bibitem{Hensen} B. Hensen et al., Experimental loophole-free violation of a Bell inequality using entangled electron
spins separated by 1.3 km. Nature (London) 526, 682 (2015)
\bibitem{Giustina} M. Giustina et al., Significant-Loophole-Free Test of Bell’s Theorem with Entangled Photons. Phys. Rev. Lett. 115, 250401 (2015)
\bibitem{Shalm} L. K. Shalm et al., Strong Loophole-Free Test of Local Realism. Phys. Rev. Lett. 115, 250402 (2015)
\end{thebibliography}

\end{document}

